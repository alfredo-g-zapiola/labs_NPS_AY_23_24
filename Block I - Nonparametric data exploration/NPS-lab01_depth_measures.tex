% Options for packages loaded elsewhere
\PassOptionsToPackage{unicode}{hyperref}
\PassOptionsToPackage{hyphens}{url}
%
\documentclass[
]{article}
\usepackage{amsmath,amssymb}
\usepackage{iftex}
\ifPDFTeX
  \usepackage[T1]{fontenc}
  \usepackage[utf8]{inputenc}
  \usepackage{textcomp} % provide euro and other symbols
\else % if luatex or xetex
  \usepackage{unicode-math} % this also loads fontspec
  \defaultfontfeatures{Scale=MatchLowercase}
  \defaultfontfeatures[\rmfamily]{Ligatures=TeX,Scale=1}
\fi
\usepackage{lmodern}
\ifPDFTeX\else
  % xetex/luatex font selection
\fi
% Use upquote if available, for straight quotes in verbatim environments
\IfFileExists{upquote.sty}{\usepackage{upquote}}{}
\IfFileExists{microtype.sty}{% use microtype if available
  \usepackage[]{microtype}
  \UseMicrotypeSet[protrusion]{basicmath} % disable protrusion for tt fonts
}{}
\makeatletter
\@ifundefined{KOMAClassName}{% if non-KOMA class
  \IfFileExists{parskip.sty}{%
    \usepackage{parskip}
  }{% else
    \setlength{\parindent}{0pt}
    \setlength{\parskip}{6pt plus 2pt minus 1pt}}
}{% if KOMA class
  \KOMAoptions{parskip=half}}
\makeatother
\usepackage{xcolor}
\usepackage[margin=1in]{geometry}
\usepackage{color}
\usepackage{fancyvrb}
\newcommand{\VerbBar}{|}
\newcommand{\VERB}{\Verb[commandchars=\\\{\}]}
\DefineVerbatimEnvironment{Highlighting}{Verbatim}{commandchars=\\\{\}}
% Add ',fontsize=\small' for more characters per line
\usepackage{framed}
\definecolor{shadecolor}{RGB}{248,248,248}
\newenvironment{Shaded}{\begin{snugshade}}{\end{snugshade}}
\newcommand{\AlertTok}[1]{\textcolor[rgb]{0.94,0.16,0.16}{#1}}
\newcommand{\AnnotationTok}[1]{\textcolor[rgb]{0.56,0.35,0.01}{\textbf{\textit{#1}}}}
\newcommand{\AttributeTok}[1]{\textcolor[rgb]{0.13,0.29,0.53}{#1}}
\newcommand{\BaseNTok}[1]{\textcolor[rgb]{0.00,0.00,0.81}{#1}}
\newcommand{\BuiltInTok}[1]{#1}
\newcommand{\CharTok}[1]{\textcolor[rgb]{0.31,0.60,0.02}{#1}}
\newcommand{\CommentTok}[1]{\textcolor[rgb]{0.56,0.35,0.01}{\textit{#1}}}
\newcommand{\CommentVarTok}[1]{\textcolor[rgb]{0.56,0.35,0.01}{\textbf{\textit{#1}}}}
\newcommand{\ConstantTok}[1]{\textcolor[rgb]{0.56,0.35,0.01}{#1}}
\newcommand{\ControlFlowTok}[1]{\textcolor[rgb]{0.13,0.29,0.53}{\textbf{#1}}}
\newcommand{\DataTypeTok}[1]{\textcolor[rgb]{0.13,0.29,0.53}{#1}}
\newcommand{\DecValTok}[1]{\textcolor[rgb]{0.00,0.00,0.81}{#1}}
\newcommand{\DocumentationTok}[1]{\textcolor[rgb]{0.56,0.35,0.01}{\textbf{\textit{#1}}}}
\newcommand{\ErrorTok}[1]{\textcolor[rgb]{0.64,0.00,0.00}{\textbf{#1}}}
\newcommand{\ExtensionTok}[1]{#1}
\newcommand{\FloatTok}[1]{\textcolor[rgb]{0.00,0.00,0.81}{#1}}
\newcommand{\FunctionTok}[1]{\textcolor[rgb]{0.13,0.29,0.53}{\textbf{#1}}}
\newcommand{\ImportTok}[1]{#1}
\newcommand{\InformationTok}[1]{\textcolor[rgb]{0.56,0.35,0.01}{\textbf{\textit{#1}}}}
\newcommand{\KeywordTok}[1]{\textcolor[rgb]{0.13,0.29,0.53}{\textbf{#1}}}
\newcommand{\NormalTok}[1]{#1}
\newcommand{\OperatorTok}[1]{\textcolor[rgb]{0.81,0.36,0.00}{\textbf{#1}}}
\newcommand{\OtherTok}[1]{\textcolor[rgb]{0.56,0.35,0.01}{#1}}
\newcommand{\PreprocessorTok}[1]{\textcolor[rgb]{0.56,0.35,0.01}{\textit{#1}}}
\newcommand{\RegionMarkerTok}[1]{#1}
\newcommand{\SpecialCharTok}[1]{\textcolor[rgb]{0.81,0.36,0.00}{\textbf{#1}}}
\newcommand{\SpecialStringTok}[1]{\textcolor[rgb]{0.31,0.60,0.02}{#1}}
\newcommand{\StringTok}[1]{\textcolor[rgb]{0.31,0.60,0.02}{#1}}
\newcommand{\VariableTok}[1]{\textcolor[rgb]{0.00,0.00,0.00}{#1}}
\newcommand{\VerbatimStringTok}[1]{\textcolor[rgb]{0.31,0.60,0.02}{#1}}
\newcommand{\WarningTok}[1]{\textcolor[rgb]{0.56,0.35,0.01}{\textbf{\textit{#1}}}}
\usepackage{graphicx}
\makeatletter
\def\maxwidth{\ifdim\Gin@nat@width>\linewidth\linewidth\else\Gin@nat@width\fi}
\def\maxheight{\ifdim\Gin@nat@height>\textheight\textheight\else\Gin@nat@height\fi}
\makeatother
% Scale images if necessary, so that they will not overflow the page
% margins by default, and it is still possible to overwrite the defaults
% using explicit options in \includegraphics[width, height, ...]{}
\setkeys{Gin}{width=\maxwidth,height=\maxheight,keepaspectratio}
% Set default figure placement to htbp
\makeatletter
\def\fps@figure{htbp}
\makeatother
\setlength{\emergencystretch}{3em} % prevent overfull lines
\providecommand{\tightlist}{%
  \setlength{\itemsep}{0pt}\setlength{\parskip}{0pt}}
\setcounter{secnumdepth}{-\maxdimen} % remove section numbering
\ifLuaTeX
  \usepackage{selnolig}  % disable illegal ligatures
\fi
\IfFileExists{bookmark.sty}{\usepackage{bookmark}}{\usepackage{hyperref}}
\IfFileExists{xurl.sty}{\usepackage{xurl}}{} % add URL line breaks if available
\urlstyle{same}
\hypersetup{
  pdftitle={Lab 01 - Depth measures},
  pdfauthor={Nonparametric statistics ay 2022/2023},
  hidelinks,
  pdfcreator={LaTeX via pandoc}}

\title{Lab 01 - Depth measures}
\author{Nonparametric statistics ay 2022/2023}
\date{2023-09-21}

\begin{document}
\maketitle

\emph{Disclaimer 1: The present material is an adaptation of the
original R script prepared by Dr.~Matteo Fontana for the a.y. 2020/2021
Nonparametric statistics course, and later Prof.~Andrea Cappozzo (a.y
2022-2023) While I acknowledge Matteo and Andrea for the (great) work
done I hereby assume responsibility for any error that may be present in
this document.}

\emph{Disclaimer 2: I will start from the assumption that you are all
intermediate R users, if this is not the case please let me know, and we
shall find a solution together.}

\hypertarget{loading-necessary-libraries}{%
\subsection{Loading necessary
libraries}\label{loading-necessary-libraries}}

\begin{Shaded}
\begin{Highlighting}[]
\FunctionTok{library}\NormalTok{(MASS)}
\FunctionTok{library}\NormalTok{(rgl)}
\FunctionTok{library}\NormalTok{(DepthProc)}
\FunctionTok{library}\NormalTok{(hexbin)}
\FunctionTok{library}\NormalTok{(aplpack)}
\FunctionTok{library}\NormalTok{(robustbase)}
\end{Highlighting}
\end{Shaded}

\hypertarget{computing-depths-with-r-some-general-ideas}{%
\subsection{Computing depths with R: some general
ideas}\label{computing-depths-with-r-some-general-ideas}}

Let us start by simulating \(500\) bivariate datapoints whose
distribution is marginally exponential

\begin{Shaded}
\begin{Highlighting}[]
\FunctionTok{set.seed}\NormalTok{(}\DecValTok{2781991}\NormalTok{) }\CommentTok{\# reproducibility}
\NormalTok{n}\OtherTok{=}\DecValTok{500}
\NormalTok{df\_bivariate\_exp }\OtherTok{=} \FunctionTok{cbind}\NormalTok{(}\FunctionTok{rexp}\NormalTok{(n), }\FunctionTok{rexp}\NormalTok{(n))}
\FunctionTok{head}\NormalTok{(df\_bivariate\_exp)}
\end{Highlighting}
\end{Shaded}

\begin{verbatim}
##           [,1]       [,2]
## [1,] 0.3004277 0.35894610
## [2,] 2.4430839 0.22392999
## [3,] 1.8444128 0.15064919
## [4,] 0.2383688 1.12290176
## [5,] 1.9613656 0.07966151
## [6,] 5.2480885 0.01762207
\end{verbatim}

and visualize their scatterplot

\begin{Shaded}
\begin{Highlighting}[]
\FunctionTok{plot}\NormalTok{(df\_bivariate\_exp[,}\DecValTok{1}\NormalTok{],df\_bivariate\_exp[,}\DecValTok{2}\NormalTok{], }\AttributeTok{xlab=}\StringTok{"exp 1"}\NormalTok{, }\AttributeTok{ylab=}\StringTok{"exp 2"}\NormalTok{)}
\end{Highlighting}
\end{Shaded}

\includegraphics{NPS-lab01_depth_measures_files/figure-latex/unnamed-chunk-4-1.pdf}

We can further employ a hexagonal binning plot to visualize the data
density

\begin{Shaded}
\begin{Highlighting}[]
\NormalTok{bin}\OtherTok{=}\FunctionTok{hexbin}\NormalTok{(df\_bivariate\_exp[,}\DecValTok{1}\NormalTok{],df\_bivariate\_exp[,}\DecValTok{2}\NormalTok{], }\AttributeTok{xbins=}\DecValTok{10}\NormalTok{, }\AttributeTok{xlab=}\StringTok{"exp 1"}\NormalTok{, }\AttributeTok{ylab=}\StringTok{"exp 2"}\NormalTok{)}
\FunctionTok{plot}\NormalTok{(bin, }\AttributeTok{main=}\StringTok{"Hexagonal Binning"}\NormalTok{) }
\end{Highlighting}
\end{Shaded}

\includegraphics{NPS-lab01_depth_measures_files/figure-latex/unnamed-chunk-5-1.pdf}

Now on depths: there are many possible packages hosted on CRAN to work
with depths in a multivariate setting. We will hereafter use
\texttt{DepthProc}, which is a good general purpose package and in
addition it has good plotting capabilities. Even though not all depth
measures introduced in class are directly available (e.g., simplicial
depth), it includes the two depth measures we will be using throughout
this section, namely \textbf{Tukey} and \textbf{Mahalanobis} depths. Let
us look at the help file for the \texttt{depth} function.

Calculating depth for a given dataset is immediately accomplished by
typing

\begin{Shaded}
\begin{Highlighting}[]
\NormalTok{tukey\_depth}\OtherTok{=}\FunctionTok{depth}\NormalTok{(}\AttributeTok{u=}\NormalTok{df\_bivariate\_exp,}\AttributeTok{method=}\StringTok{\textquotesingle{}Tukey\textquotesingle{}}\NormalTok{)}
\end{Highlighting}
\end{Shaded}

It may be useful (spoiler alert, it will become apparent why when we
study onparametric forecasting) to calculate the depth of a point
relative to a sample. You can do it by:

\begin{Shaded}
\begin{Highlighting}[]
\FunctionTok{depth}\NormalTok{(}\AttributeTok{u =} \FunctionTok{c}\NormalTok{(}\DecValTok{0}\NormalTok{, }\DecValTok{0}\NormalTok{), }\AttributeTok{X =}\NormalTok{ df\_bivariate\_exp, }\AttributeTok{method =} \StringTok{\textquotesingle{}Tukey\textquotesingle{}}\NormalTok{) }
\end{Highlighting}
\end{Shaded}

\begin{verbatim}
## Depth method:  Tukey 
## [1] 0
\end{verbatim}

Compute the median (deepest point) with \texttt{depthMedian} function

\begin{Shaded}
\begin{Highlighting}[]
\FunctionTok{depthMedian}\NormalTok{(df\_bivariate\_exp,}\AttributeTok{depth\_params =} \FunctionTok{list}\NormalTok{(}\AttributeTok{method=}\StringTok{\textquotesingle{}Tukey\textquotesingle{}}\NormalTok{))}
\end{Highlighting}
\end{Shaded}

\begin{verbatim}
## [1] 0.8364350 0.9297927
\end{verbatim}

Or, having already computed the Tukey depth for the entire sample:

\begin{Shaded}
\begin{Highlighting}[]
\NormalTok{df\_bivariate\_exp[}\FunctionTok{which.max}\NormalTok{(tukey\_depth),]}
\end{Highlighting}
\end{Shaded}

\begin{verbatim}
## [1] 0.8364350 0.9297927
\end{verbatim}

If you have the luck of dealing with a bivariate dataset, you can easily
visualize the depth surface in a very convenient way. \texttt{DepthProc}
offers you two possible methods:

\begin{Shaded}
\begin{Highlighting}[]
\FunctionTok{depthContour}\NormalTok{(df\_bivariate\_exp,}\AttributeTok{depth\_params =} \FunctionTok{list}\NormalTok{(}\AttributeTok{method=}\StringTok{\textquotesingle{}Tukey\textquotesingle{}}\NormalTok{))}
\end{Highlighting}
\end{Shaded}

\includegraphics{NPS-lab01_depth_measures_files/figure-latex/unnamed-chunk-10-1.pdf}

or

\begin{Shaded}
\begin{Highlighting}[]
\FunctionTok{depthPersp}\NormalTok{(df\_bivariate\_exp,}\AttributeTok{depth\_params =} \FunctionTok{list}\NormalTok{(}\AttributeTok{method=}\StringTok{\textquotesingle{}Tukey\textquotesingle{}}\NormalTok{))}
\end{Highlighting}
\end{Shaded}

\includegraphics{NPS-lab01_depth_measures_files/figure-latex/unnamed-chunk-11-1.pdf}

For additional special effects:

\begin{Shaded}
\begin{Highlighting}[]
\FunctionTok{depthPersp}\NormalTok{(df\_bivariate\_exp,}\AttributeTok{depth\_params =} \FunctionTok{list}\NormalTok{(}\AttributeTok{method=}\StringTok{\textquotesingle{}Tukey\textquotesingle{}}\NormalTok{),}\AttributeTok{plot\_method =} \StringTok{\textquotesingle{}rgl\textquotesingle{}}\NormalTok{)}
\end{Highlighting}
\end{Shaded}

/tmp/RtmpOtg4JP/file11fce70240502.png \#\#\#\#\# Computational caveat
Since the Tukey depth is obtained by counting points in the space of all
hyperplanes(surfaces) in \(\mathbb{R}^d\), when \(d>2\), the
computational burden increases.

\begin{Shaded}
\begin{Highlighting}[]
\CommentTok{\# generate data as before}
\NormalTok{n }
\end{Highlighting}
\end{Shaded}

\begin{verbatim}
## [1] 500
\end{verbatim}

\begin{Shaded}
\begin{Highlighting}[]
\NormalTok{df\_quatrivariate\_exp }\OtherTok{=} \FunctionTok{cbind}\NormalTok{(}\FunctionTok{rexp}\NormalTok{(n), }\FunctionTok{rexp}\NormalTok{(n), }\FunctionTok{rexp}\NormalTok{(n), }\FunctionTok{rexp}\NormalTok{(n), }\FunctionTok{rexp}\NormalTok{(n))}
\CommentTok{\# obtain tukey depth}
\FunctionTok{depthMedian}\NormalTok{(df\_quatrivariate\_exp,}\AttributeTok{depth\_params =} \FunctionTok{list}\NormalTok{(}\AttributeTok{method=}\StringTok{\textquotesingle{}Tukey\textquotesingle{}}\NormalTok{))}
\end{Highlighting}
\end{Shaded}

\begin{verbatim}
## [1] 0.5861941 0.7504385 0.8633159 0.6090142 0.9498993
\end{verbatim}

The very same analysis can be carried out by considering a different
depth measure:

\begin{Shaded}
\begin{Highlighting}[]
\NormalTok{maha\_depth }\OtherTok{\textless{}{-}} \FunctionTok{depth}\NormalTok{(df\_bivariate\_exp,}\AttributeTok{method=}\StringTok{\textquotesingle{}Mahalanobis\textquotesingle{}}\NormalTok{) }
\end{Highlighting}
\end{Shaded}

This can be easily hard coded, recall the definition you have seen in
class:

\[{M}_{h}D({F;X} ^ {n}) = \frac{ 1 }{ 1 + {{(x - \mu_F)} ^ {T}}{{\Sigma} ^ {-1}}(x - \mu_F) }\]

\begin{Shaded}
\begin{Highlighting}[]
\NormalTok{sample\_mean }\OtherTok{\textless{}{-}} \FunctionTok{colMeans}\NormalTok{(df\_bivariate\_exp)}
\NormalTok{sample\_S }\OtherTok{\textless{}{-}} \FunctionTok{cov}\NormalTok{(df\_bivariate\_exp)}

\NormalTok{maha\_depth\_manual }\OtherTok{\textless{}{-}} \DecValTok{1}\SpecialCharTok{/}\NormalTok{(}\DecValTok{1}\SpecialCharTok{+}\FunctionTok{mahalanobis}\NormalTok{(}\AttributeTok{x =}\NormalTok{ df\_bivariate\_exp,}\AttributeTok{center =}\NormalTok{ sample\_mean,}\AttributeTok{cov =}\NormalTok{ sample\_S))}
\end{Highlighting}
\end{Shaded}

And check that the obtained result is equal to the one obtained via the
\texttt{depth} function

\begin{Shaded}
\begin{Highlighting}[]
\FunctionTok{all}\NormalTok{(}\FunctionTok{abs}\NormalTok{(maha\_depth}\SpecialCharTok{{-}}\NormalTok{maha\_depth\_manual)}\SpecialCharTok{\textless{}}\FloatTok{1e{-}15}\NormalTok{) }\CommentTok{\# food for thought: sqrt(2) \^{} 2 == 2?}
\end{Highlighting}
\end{Shaded}

\begin{verbatim}
## [1] TRUE
\end{verbatim}

\begin{Shaded}
\begin{Highlighting}[]
\FunctionTok{depthMedian}\NormalTok{(df\_bivariate\_exp,}\AttributeTok{depth\_params =} \FunctionTok{list}\NormalTok{(}\AttributeTok{method=}\StringTok{\textquotesingle{}Mahalanobis\textquotesingle{}}\NormalTok{))}
\end{Highlighting}
\end{Shaded}

\begin{verbatim}
## [1] 0.9211997 1.0228290
\end{verbatim}

\begin{Shaded}
\begin{Highlighting}[]
\NormalTok{df\_bivariate\_exp[}\FunctionTok{which.max}\NormalTok{(maha\_depth\_manual),]}
\end{Highlighting}
\end{Shaded}

\begin{verbatim}
## [1] 0.9211997 1.0228290
\end{verbatim}

And again the graphical outputs:

\begin{Shaded}
\begin{Highlighting}[]
\FunctionTok{depthContour}\NormalTok{(df\_bivariate\_exp,}\AttributeTok{depth\_params =} \FunctionTok{list}\NormalTok{(}\AttributeTok{method=}\StringTok{\textquotesingle{}Mahalanobis\textquotesingle{}}\NormalTok{))}
\end{Highlighting}
\end{Shaded}

\includegraphics{NPS-lab01_depth_measures_files/figure-latex/unnamed-chunk-18-1.pdf}

\begin{Shaded}
\begin{Highlighting}[]
\FunctionTok{depthPersp}\NormalTok{(df\_bivariate\_exp,}\AttributeTok{depth\_params =} \FunctionTok{list}\NormalTok{(}\AttributeTok{method=}\StringTok{\textquotesingle{}Mahalanobis\textquotesingle{}}\NormalTok{))}
\end{Highlighting}
\end{Shaded}

\includegraphics{NPS-lab01_depth_measures_files/figure-latex/unnamed-chunk-18-2.pdf}

\begin{Shaded}
\begin{Highlighting}[]
\FunctionTok{depthPersp}\NormalTok{(df\_bivariate\_exp,}\AttributeTok{depth\_params =} \FunctionTok{list}\NormalTok{(}\AttributeTok{method=}\StringTok{\textquotesingle{}Mahalanobis\textquotesingle{}}\NormalTok{),}\AttributeTok{plot\_method =} \StringTok{\textquotesingle{}rgl\textquotesingle{}}\NormalTok{)}
\end{Highlighting}
\end{Shaded}

/tmp/RtmpOtg4JP/file11fce335b3fa0.png

Please note that anything that comes out of \texttt{depthContour} or
\texttt{depthPersp} is NOT a density: we are not doing density
estimation here, we are performing data exploration by means of a
nonparametric procedure.

\hypertarget{exercise}{%
\subsubsection{Exercise}\label{exercise}}

Try out the routines seen so far on something even more exotic:

\begin{Shaded}
\begin{Highlighting}[]
\FunctionTok{set.seed}\NormalTok{(}\DecValTok{1992}\NormalTok{)}
\NormalTok{df\_bivariate\_cauchy }\OtherTok{=} \FunctionTok{cbind}\NormalTok{(}\FunctionTok{rcauchy}\NormalTok{(n,}\AttributeTok{location=}\DecValTok{0}\NormalTok{,}\AttributeTok{scale=}\NormalTok{.}\DecValTok{001}\NormalTok{), }\FunctionTok{rcauchy}\NormalTok{(n,}\AttributeTok{location =} \DecValTok{0}\NormalTok{,}\AttributeTok{scale=}\NormalTok{.}\DecValTok{001}\NormalTok{))}
\FunctionTok{head}\NormalTok{(df\_bivariate\_cauchy)}
\end{Highlighting}
\end{Shaded}

\begin{verbatim}
##               [,1]         [,2]
## [1,] -0.0024184800 0.0008798105
## [2,] -0.0006282804 0.0009932694
## [3,] -0.0006975300 0.0010347342
## [4,] -0.0038080204 0.0141419885
## [5,]  0.0061098712 0.0014929490
## [6,] -0.0012696290 0.0008865863
\end{verbatim}

The Cauchy distribution is a distribution that has neither the first
moment (the mean) nor the second moment (the variance). This means that
the CLT does not apply.

\begin{itemize}
\tightlist
\item
  Can we still perform non-parametric data exploration with depth
  measures?
\item
  Would the Mahalanobis depth be a sensible measure to be used in this
  context? Why not?
\end{itemize}

\hypertarget{multivariate-outlier-detection-via-depth-measures}{%
\subsection{Multivariate outlier detection via depth
measures}\label{multivariate-outlier-detection-via-depth-measures}}

In the previous Section we have computed depth measures for a given
dataset and we have seen how to effectively plot them. This resulted in
a convenient way to nonparametrically explore a (bivariate) dataset.
Nevertheless, one of the main aims for a statistician to employ depth
measures is to perform multivariate outlier detection. To appreciate
this, let us simulate \(100\) data points, of which \(95\%\) comes from
a multivariate normal with mean vector \texttt{mu\_good} and covariance
matrix \texttt{sigma\_common} and the other 5\% from another
multivariate normal, which we assume is our outlier generator process.

\begin{Shaded}
\begin{Highlighting}[]
\NormalTok{mu\_good }\OtherTok{=} \FunctionTok{c}\NormalTok{(}\DecValTok{0}\NormalTok{,}\DecValTok{0}\NormalTok{) }
\NormalTok{mu\_outliers }\OtherTok{=} \FunctionTok{c}\NormalTok{(}\DecValTok{7}\NormalTok{,}\DecValTok{7}\NormalTok{)}

\NormalTok{sigma\_common }\OtherTok{=} \FunctionTok{matrix}\NormalTok{(}\FunctionTok{c}\NormalTok{(}\DecValTok{1}\NormalTok{,.}\DecValTok{7}\NormalTok{,.}\DecValTok{7}\NormalTok{,}\DecValTok{1}\NormalTok{), }\AttributeTok{ncol =} \DecValTok{2}\NormalTok{)}

\NormalTok{frac }\OtherTok{=}\NormalTok{ .}\DecValTok{05}
\NormalTok{n}\OtherTok{=}\DecValTok{100}
\CommentTok{\# sample points}
\NormalTok{n\_good}\OtherTok{=}\FunctionTok{ceiling}\NormalTok{(n}\SpecialCharTok{*}\NormalTok{(}\DecValTok{1}\SpecialCharTok{{-}}\NormalTok{frac))}
\NormalTok{n\_outliers}\OtherTok{=}\NormalTok{n}\SpecialCharTok{{-}}\NormalTok{n\_good}
\NormalTok{df\_contaminated\_normals }\OtherTok{=} \FunctionTok{data.frame}\NormalTok{(}\FunctionTok{rbind}\NormalTok{(}
  \FunctionTok{mvrnorm}\NormalTok{(n\_good, mu\_good, sigma\_common),}
  \FunctionTok{mvrnorm}\NormalTok{(n\_outliers, mu\_outliers, sigma\_common)}
\NormalTok{))}
\end{Highlighting}
\end{Shaded}

Let us visualize the true nature of our dataset

\begin{Shaded}
\begin{Highlighting}[]
\NormalTok{class }\OtherTok{\textless{}{-}} \FunctionTok{c}\NormalTok{(}\FunctionTok{rep}\NormalTok{(}\DecValTok{1}\NormalTok{,n\_good),}\FunctionTok{rep}\NormalTok{(}\DecValTok{2}\NormalTok{,n\_outliers))}
\FunctionTok{plot}\NormalTok{(df\_contaminated\_normals,}\AttributeTok{xlab=}\StringTok{"Norm 1"}\NormalTok{, }\AttributeTok{ylab=}\StringTok{"Norm 2"}\NormalTok{,}\AttributeTok{col=}\NormalTok{class)}
\end{Highlighting}
\end{Shaded}

\includegraphics{NPS-lab01_depth_measures_files/figure-latex/unnamed-chunk-21-1.pdf}

We can clearly see those red points up there\ldots{} how can we flag
them in an automated fashion? The depth contour plot surely helps

\begin{Shaded}
\begin{Highlighting}[]
\FunctionTok{depthContour}\NormalTok{(}
\NormalTok{  df\_contaminated\_normals,}
  \AttributeTok{depth\_params =} \FunctionTok{list}\NormalTok{(}\AttributeTok{method =} \StringTok{\textquotesingle{}Tukey\textquotesingle{}}\NormalTok{),}
  \AttributeTok{points =} \ConstantTok{TRUE}\NormalTok{,}
  \AttributeTok{colors =} \FunctionTok{colorRampPalette}\NormalTok{(}\FunctionTok{c}\NormalTok{(}\StringTok{\textquotesingle{}white\textquotesingle{}}\NormalTok{, }\StringTok{\textquotesingle{}navy\textquotesingle{}}\NormalTok{)),}
  \AttributeTok{levels =} \DecValTok{10}\NormalTok{,}
  \AttributeTok{pdmedian =}\NormalTok{ F,}
  \AttributeTok{graph\_params =} \FunctionTok{list}\NormalTok{(}\AttributeTok{cex=}\NormalTok{.}\DecValTok{01}\NormalTok{, }\AttributeTok{pch=}\DecValTok{1}\NormalTok{),}
  \AttributeTok{pmean =}\NormalTok{ F}
\NormalTok{)}
\end{Highlighting}
\end{Shaded}

\includegraphics{NPS-lab01_depth_measures_files/figure-latex/unnamed-chunk-22-1.pdf}

But, as we have seen in class, a very handy graphical tool for spotting
multivariate outliers is the bagplot. The \texttt{bagplot} function from
the \texttt{aplpack} package can be used to display bagplots for
bivariate data

\begin{Shaded}
\begin{Highlighting}[]
\FunctionTok{bagplot}\NormalTok{(df\_contaminated\_normals, }\AttributeTok{factor =} \DecValTok{3}\NormalTok{)}
\end{Highlighting}
\end{Shaded}

\includegraphics{NPS-lab01_depth_measures_files/figure-latex/unnamed-chunk-23-1.pdf}

If we check the help page for the \texttt{bagplot} function, we see that
we have a lot of room for customization:

\begin{Shaded}
\begin{Highlighting}[]
\NormalTok{aplpack}\SpecialCharTok{::}\FunctionTok{bagplot}\NormalTok{(df\_contaminated\_normals,}\AttributeTok{show.whiskers =}\NormalTok{ F,}\AttributeTok{main=}\StringTok{"Bagplot"}\NormalTok{)}
\end{Highlighting}
\end{Shaded}

\includegraphics{NPS-lab01_depth_measures_files/figure-latex/unnamed-chunk-24-1.pdf}

\begin{Shaded}
\begin{Highlighting}[]
\NormalTok{aplpack}\SpecialCharTok{::}\FunctionTok{bagplot}\NormalTok{(df\_contaminated\_normals,}\AttributeTok{show.loophull =}\NormalTok{ F,}\AttributeTok{main=}\StringTok{"Sunburst plot"}\NormalTok{)}
\end{Highlighting}
\end{Shaded}

\includegraphics{NPS-lab01_depth_measures_files/figure-latex/unnamed-chunk-24-2.pdf}

In addition, if we save the output of the bagplot to an object, we can
automatically extract the outliers

\begin{Shaded}
\begin{Highlighting}[]
\NormalTok{bagplot\_cont\_normals }\OtherTok{\textless{}{-}} \FunctionTok{bagplot}\NormalTok{(df\_contaminated\_normals)}
\end{Highlighting}
\end{Shaded}

\includegraphics{NPS-lab01_depth_measures_files/figure-latex/unnamed-chunk-25-1.pdf}

\begin{Shaded}
\begin{Highlighting}[]
\NormalTok{outlying\_obs }\OtherTok{\textless{}{-}}\NormalTok{ bagplot\_cont\_normals}\SpecialCharTok{$}\NormalTok{pxy.outlier}
\end{Highlighting}
\end{Shaded}

Once the outlying units have been identified, one can discard them from
the original data and keep working on the clean subset only. There are
several ways to do this.

A more ``sql-oriented'' approach:

\begin{Shaded}
\begin{Highlighting}[]
\NormalTok{df\_clean\_1 }\OtherTok{\textless{}{-}}
\NormalTok{  dplyr}\SpecialCharTok{::}\FunctionTok{anti\_join}\NormalTok{(}
    \AttributeTok{x =}\NormalTok{ df\_contaminated\_normals,}
    \AttributeTok{y =}\NormalTok{ outlying\_obs,}
    \AttributeTok{by =} \FunctionTok{c}\NormalTok{(}\StringTok{"X1"} \OtherTok{=} \StringTok{"x"}\NormalTok{, }\StringTok{"X2"} \OtherTok{=} \StringTok{"y"}\NormalTok{),}
    \AttributeTok{copy =} \ConstantTok{TRUE}
\NormalTok{  )}
\end{Highlighting}
\end{Shaded}

A more ``object-oriented programming'' approach:

\begin{Shaded}
\begin{Highlighting}[]
\NormalTok{ind\_outlying\_obs }\OtherTok{\textless{}{-}} \FunctionTok{which}\NormalTok{(}\FunctionTok{apply}\NormalTok{(df\_contaminated\_normals,}\DecValTok{1}\NormalTok{,}\ControlFlowTok{function}\NormalTok{(x) }\FunctionTok{all}\NormalTok{(x }\SpecialCharTok{\%in\%}\NormalTok{ outlying\_obs)))}
\NormalTok{df\_clean\_2 }\OtherTok{\textless{}{-}}\NormalTok{ df\_contaminated\_normals[}\SpecialCharTok{{-}}\NormalTok{ind\_outlying\_obs,]}
\end{Highlighting}
\end{Shaded}

\begin{Shaded}
\begin{Highlighting}[]
\FunctionTok{all.equal}\NormalTok{(df\_clean\_1,df\_clean\_2)}
\end{Highlighting}
\end{Shaded}

\begin{verbatim}
## [1] TRUE
\end{verbatim}

\hypertarget{outlier-detection-in-star-cluster-cyg-ob1-data}{%
\subsection{Outlier detection in Star Cluster CYG OB1
data}\label{outlier-detection-in-star-cluster-cyg-ob1-data}}

Data for the Hertzsprung-Russell Diagram of the Star Cluster CYG OB1,
which contains \(47\) stars in the direction of Cygnus, from C.Doom.

\begin{itemize}
\tightlist
\item
  The first variable is the logarithm of the effective temperature at
  the surface of the star (Te)
\item
  The second one is the logarithm of its light intensity (L/L0). The
  Hertzsprung-Russell diagram is the scatterplot of these data points,
  where the log temperature is plotted from left to right.
\end{itemize}

\begin{Shaded}
\begin{Highlighting}[]
\FunctionTok{data}\NormalTok{(starsCYG, }\AttributeTok{package =} \StringTok{"robustbase"}\NormalTok{)}
\FunctionTok{names}\NormalTok{(starsCYG)}
\end{Highlighting}
\end{Shaded}

\begin{verbatim}
## [1] "log.Te"    "log.light"
\end{verbatim}

\begin{Shaded}
\begin{Highlighting}[]
\FunctionTok{plot}\NormalTok{(starsCYG, }\AttributeTok{main=}\StringTok{"Star Cluster CYG OB1"}\NormalTok{)}
\end{Highlighting}
\end{Shaded}

\includegraphics{NPS-lab01_depth_measures_files/figure-latex/unnamed-chunk-29-1.pdf}

We can see two groups of points: the majority which tends to follow a
steep band, the so called Main Sequence, and four stars in the
upper-left corner. In astronomy the \(43\) stars are said to lie on the
Main sequence and the four remaining stars are the red giants, namely
points with indexes 11, 20, 30 and 34. In details, the red giants are
very bright, but they emit light with a very low color-temperature (and
thus their surface temperature is still fairly low). We can easily
isolate them thanks to the procedure seen so far. Let us look at the
perspective plot of the depth surface

\begin{Shaded}
\begin{Highlighting}[]
\FunctionTok{depthContour}\NormalTok{(}\FunctionTok{as.matrix}\NormalTok{(starsCYG), }\AttributeTok{depth\_params =} \FunctionTok{list}\NormalTok{(}\AttributeTok{method=}\StringTok{\textquotesingle{}Tukey\textquotesingle{}}\NormalTok{), }\AttributeTok{points=}\ConstantTok{TRUE}\NormalTok{)}
\end{Highlighting}
\end{Shaded}

\includegraphics{NPS-lab01_depth_measures_files/figure-latex/unnamed-chunk-30-1.pdf}

As you can see, the sample mean is biased due to the presence of the 4
red giants, whereas the Tukey Median is not. Let us compute it:

\begin{Shaded}
\begin{Highlighting}[]
\FunctionTok{depthMedian}\NormalTok{(starsCYG) }
\end{Highlighting}
\end{Shaded}

\begin{verbatim}
##    log.Te log.light 
##      4.38      5.02
\end{verbatim}

As before, we can use the bagplot to visualize and flag the outlying
data points.

\begin{Shaded}
\begin{Highlighting}[]
\NormalTok{bagplot\_starsCYG }\OtherTok{\textless{}{-}} \FunctionTok{with}\NormalTok{(starsCYG,aplpack}\SpecialCharTok{::}\FunctionTok{bagplot}\NormalTok{(log.Te,log.light))}
\end{Highlighting}
\end{Shaded}

\includegraphics{NPS-lab01_depth_measures_files/figure-latex/unnamed-chunk-32-1.pdf}

\begin{Shaded}
\begin{Highlighting}[]
\NormalTok{red\_giants }\OtherTok{\textless{}{-}}\NormalTok{ bagplot\_starsCYG}\SpecialCharTok{$}\NormalTok{pxy.outlier}
\NormalTok{ind\_outlying\_obs }\OtherTok{\textless{}{-}} \FunctionTok{which}\NormalTok{(}\FunctionTok{apply}\NormalTok{(starsCYG,}\DecValTok{1}\NormalTok{,}\ControlFlowTok{function}\NormalTok{(x) }\FunctionTok{all}\NormalTok{(x }\SpecialCharTok{\%in\%}\NormalTok{ red\_giants)))}
\NormalTok{ind\_outlying\_obs}
\end{Highlighting}
\end{Shaded}

\begin{verbatim}
## [1] 11 20 30 34
\end{verbatim}

\hypertarget{moving-beyond-r-2}{%
\subsection{\texorpdfstring{Moving beyond
\(R ^2\)}{Moving beyond R \^{}2}}\label{moving-beyond-r-2}}

So far, we have only dealt with bivariate data. Even though the
computational complexity escalates quickly when it comes to compute
depth measures in \(R ^d\), with \(d>2\), we can still appreciate their
usefulness in moderate dimension. Let us generate a trivariate dataset
with contamination.

\begin{Shaded}
\begin{Highlighting}[]
\NormalTok{mu\_good }\OtherTok{=} \FunctionTok{rep}\NormalTok{(}\DecValTok{0}\NormalTok{,}\DecValTok{3}\NormalTok{)}
\NormalTok{mu\_outliers }\OtherTok{=} \FunctionTok{c}\NormalTok{(}\DecValTok{12}\NormalTok{,}\DecValTok{12}\NormalTok{,}\DecValTok{3}\NormalTok{)}

\NormalTok{sigma\_common }\OtherTok{=} \FunctionTok{diag}\NormalTok{(}\DecValTok{3}\NormalTok{)}\SpecialCharTok{*}\DecValTok{2}

\NormalTok{frac }\OtherTok{=}\NormalTok{ .}\DecValTok{1}
\NormalTok{n}\OtherTok{=}\DecValTok{300}
\CommentTok{\# sample points}
\NormalTok{n\_good}\OtherTok{=}\FunctionTok{ceiling}\NormalTok{(n}\SpecialCharTok{*}\NormalTok{(}\DecValTok{1}\SpecialCharTok{{-}}\NormalTok{frac))}
\NormalTok{n\_outliers}\OtherTok{=}\NormalTok{n}\SpecialCharTok{{-}}\NormalTok{n\_good}
\NormalTok{df\_3 }\OtherTok{=} \FunctionTok{data.frame}\NormalTok{(}\FunctionTok{rbind}\NormalTok{(}
  \FunctionTok{mvrnorm}\NormalTok{(n\_good, mu\_good, sigma\_common),}
  \FunctionTok{mvrnorm}\NormalTok{(n\_outliers, mu\_outliers, sigma\_common)}
\NormalTok{))}
\NormalTok{class }\OtherTok{\textless{}{-}} \FunctionTok{c}\NormalTok{(}\FunctionTok{rep}\NormalTok{(}\DecValTok{1}\NormalTok{,n\_good),}\FunctionTok{rep}\NormalTok{(}\DecValTok{2}\NormalTok{,n\_outliers))}
\FunctionTok{pairs}\NormalTok{(df\_3, }\AttributeTok{col=}\NormalTok{class)}
\end{Highlighting}
\end{Shaded}

\includegraphics{NPS-lab01_depth_measures_files/figure-latex/unnamed-chunk-33-1.pdf}

To visualize the outliers in this context we retort to a bagplot matrix:

\begin{Shaded}
\begin{Highlighting}[]
\NormalTok{bagplot\_matrix }\OtherTok{\textless{}{-}}\NormalTok{ aplpack}\SpecialCharTok{::}\FunctionTok{bagplot.pairs}\NormalTok{(df\_3)}
\end{Highlighting}
\end{Shaded}

\includegraphics{NPS-lab01_depth_measures_files/figure-latex/unnamed-chunk-34-1.pdf}

You can notice that outlier detection becomes more difficult when the
dimension increases, as

\begin{itemize}
\tightlist
\item
  some outliers may be wrongly flagged as genuine points
\item
  some good points may be wrongly flagged as outliers
\end{itemize}

These phenomena are respectively denoted as \textbf{masking} and
\textbf{swamping}, we will cover it in details during the robust
statistics module of this course!

Let us conclude this lab session by looking at some DD-plots. For two
probability distributions \(F\) and \(G\) , both in \(R ^d\), and
\(D(\cdot)\) an affine-invariant depth, we can define depth vs.~depth
plot being very useful generalization of the one dimensional
quantile-quantile plot:

\[
D D(F, G)=\left\{\left(D_{F}(x), D_{G}(x)\right) \text { for all } x \in \mathbb{R}^{d}\right\}
\]

Its sample counterpart calculated for two samples
\(\mathbf{X} =\left\{X_1, \cdots, X_n \right\}\) from \(F\), and
\(\mathbf{Y} =\left\{Y_1, \cdots, Y_m \right\}\) from \(G\) is defined
as

\[
D D\left(F_{n}, G_{m}\right)=\left\{\left(D_{F_{n}}(z), D_{G_{m}}(z)\right), z \in\{\mathbf{X} \cup \mathbf{Y}\}\right\}
\]

The \texttt{ddPlot} function automatically computes and plots it:

\begin{Shaded}
\begin{Highlighting}[]
\NormalTok{df\_good }\OtherTok{\textless{}{-}}\NormalTok{ df\_3[}\DecValTok{1}\SpecialCharTok{:}\NormalTok{n\_good,]}
\NormalTok{df\_out }\OtherTok{\textless{}{-}}\NormalTok{ df\_3[(n\_good}\SpecialCharTok{+}\DecValTok{1}\NormalTok{)}\SpecialCharTok{:}\NormalTok{n,]}
\FunctionTok{ddPlot}\NormalTok{(}\AttributeTok{x =}\NormalTok{ df\_good,}\AttributeTok{y =}\NormalTok{ df\_out,}\AttributeTok{depth\_params =} \FunctionTok{list}\NormalTok{(}\AttributeTok{method=}\StringTok{\textquotesingle{}Tukey\textquotesingle{}}\NormalTok{))}
\end{Highlighting}
\end{Shaded}

\begin{verbatim}
## DDPlot
\end{verbatim}

\includegraphics{NPS-lab01_depth_measures_files/figure-latex/unnamed-chunk-35-1.pdf}

\begin{verbatim}
## 
## Depth Metohod:
##   Tukey
\end{verbatim}

They would line up if they came from the same distribution

It is easy to manually build a DD-plot

\begin{Shaded}
\begin{Highlighting}[]
\NormalTok{depth\_good }\OtherTok{\textless{}{-}} \FunctionTok{depth}\NormalTok{(}\AttributeTok{u =}\NormalTok{ df\_3,}\AttributeTok{X =}\NormalTok{ df\_good,}\AttributeTok{method =} \StringTok{"Tukey"}\NormalTok{)}
\NormalTok{depth\_out }\OtherTok{\textless{}{-}} \FunctionTok{depth}\NormalTok{(}\AttributeTok{u =}\NormalTok{ df\_3,}\AttributeTok{X =}\NormalTok{ df\_out,}\AttributeTok{method =} \StringTok{"Tukey"}\NormalTok{)}
\FunctionTok{plot}\NormalTok{(depth\_good,depth\_out, }\AttributeTok{col=}\StringTok{"blue"}\NormalTok{, }\AttributeTok{xlab=}\StringTok{"X depth"}\NormalTok{, }\AttributeTok{ylab=}\StringTok{"Y depth"}\NormalTok{, }\AttributeTok{main=} \StringTok{"Depth vs. depth plot"}\NormalTok{)}
\FunctionTok{grid}\NormalTok{(}\DecValTok{10}\NormalTok{, }\DecValTok{10}\NormalTok{, }\AttributeTok{col=}\StringTok{"grey50"}\NormalTok{, }\AttributeTok{lty=}\DecValTok{1}\NormalTok{)}
\FunctionTok{abline}\NormalTok{(}\DecValTok{0}\NormalTok{,}\DecValTok{1}\NormalTok{, }\AttributeTok{col=}\StringTok{"grey50"}\NormalTok{)}
\end{Highlighting}
\end{Shaded}

\includegraphics{NPS-lab01_depth_measures_files/figure-latex/unnamed-chunk-36-1.pdf}

Clearly the distributions are not the same\ldots{} What if we had some
extra samples coming from \(F\) (e.g., the trivariate normal centered at
\(0\))?

\begin{Shaded}
\begin{Highlighting}[]
\NormalTok{n\_extra }\OtherTok{\textless{}{-}} \DecValTok{100}
\NormalTok{df\_extra }\OtherTok{\textless{}{-}} \FunctionTok{data.frame}\NormalTok{(}\FunctionTok{mvrnorm}\NormalTok{(n\_extra, mu\_good, sigma\_common))}
\FunctionTok{ddPlot}\NormalTok{(}\AttributeTok{x =}\NormalTok{ df\_extra, df\_good,}\AttributeTok{depth\_params=}\FunctionTok{list}\NormalTok{(}\AttributeTok{method=}\StringTok{\textquotesingle{}Tukey\textquotesingle{}}\NormalTok{))}
\end{Highlighting}
\end{Shaded}

\begin{verbatim}
## DDPlot
\end{verbatim}

\includegraphics{NPS-lab01_depth_measures_files/figure-latex/unnamed-chunk-37-1.pdf}

\begin{verbatim}
## 
## Depth Metohod:
##   Tukey
\end{verbatim}

Indeed this time we can conclude the same distribution generated the
data.

\end{document}
